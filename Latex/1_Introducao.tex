\chapter{Introdução}\label{cap:Intro}

Neste capítulo introdutório tem como finalidade a apresentação do tema, os objetivos do trabalho realizado e relata-se a sequência em que foi desenvolvido. Primeiramente, a situação problema será apresentada. Em seguida, as principais motivações e objetivos propostos neste trabalho. Após esses pontos e para finalizar, a estrutura dos capítulos dessa dissertação será apresentada.

\section{Definição do Problema}

A realização do controle do pêndulo invertido, seja qual for sua variação, é um dos exemplos mais importantes na teoria de controle moderno. Sua estrutura resulta em um sistema não linear bastante complexo. Isso dificulta bastante a utilização de técnicas clássicas de teoria de controle.

O controle de sistemas não linear é uma técnica na qual um procedimento ou processo é conseguido sem a interferência humana. Para realizar a mesma, é preciso que um programa de instruções seja escrito  combinado a um sistema de controle que executa as tarefas. De acordo com \cite{Groover:11}, um processo automatizado necessita de energia não só para operar a condução do processo como para operar o programa e o sistema de controle. Ademais, considera-se uma grande aliada da otimização do desempenho, já que com essa tecnologia é possível conhecer indicadores que auxiliam a gestão, acelera os processos e remove trabalhos repetitivos, dispensáveis e não aceitáveis ergonomicamente.

Uma grande área em ascensão da robótica é a robótica móvel. Sendo assim, os robôs se locomovem dentro de um ambiente qualquer de diversas formas, com rodas, esteiras, pernas mecânicas, dentre outras. Estes robôs são caracterizados pela capacidade de se deslocar, podendo ser de modo guiado, semiautônomo ou totalmente autônomo \citep{Jung:05}. Algumas de suas mais variadas aplicações são: aspiração de pó, entrega de alimentos, vigilância predial, busca e salvamento, etc.

Assim sendo, neste trabalho, realizou-se o desenvolvimento de um veículo do tipo pêndulo invertido sobre duas rodas, que pode ser classificado como um robô do tipo móvel. Para a estabilização na vertical deste sistema, utilizou-se de técnicas de controle robustas que irão conseguir tratar as simplificações e considerações de seu modelo matemático. 

\section{Motivação}
O estudo da Teoria de Controle em sistemas dinâmicos e a implementação de controladores em uma planta física, complexa e com características não lineares, além da concretização de um projeto completo é de suma importância na consolidação dos conceitos assimilados ao longo da formação acadêmica de um profissional da área de Mecatrônica.

Além de tudo, após o estudo de três disciplinas seguidas da área de controle e a vontade do autor de conseguir realizar o controle de uma planta clássica, porém com um nível de dificuldade elevada bem como a construção de um projeto puramente mecatrônico que aborda as quatro grandes áreas: mecânica, controle, computação e eletrônica.

\section{Objetivos do Trabalho}
Aqui é descrito de forma sucinta o objetivo geral e os objetivos específicos do trabalho. Fazendo-se cumprir esses objetivos específicos, espera-se alcançar o objetivo geral.

\subsection{Objetivo Geral}
Construir um sistema denominado pêndulo invertido sobre duas rodas e controlar a posição de sua estrutura para que a mesma fique paralela com a vertical (0$^\circ$ ou 0 radianos) utilizando controlador ótimo.

\subsection{Objetivos específicos}
Listam-se os seguintes objetivos específicos para o projeto completo:
\begin{itemize}
  \item desenvolver por meio do SolidWorks o conceito do protótipo;
  \item levantar a lista e o custo dos materiais;
  \item obter o modelo não linear do sistema e linearizar a partir de técnicas jacobianas;
  \item discretizar o modelo linear obtido;
  \item projetar o controlador LQG com base no modelo linear;
  \item avaliar a resposta do controlador aplicando ao modelo não linear;
  \item realizar a construção da planta física;
  \item realizar a calibração do sensor e testes nos componentes eletrônicos da planta;
  \item validar o modelo obtido;
  \item implementar/embarcar o controlador no sistema.
\end{itemize}


\section{Organização do Documento}

Este documento é dividido em quatro capítulos além das considerações finais. 

O presente capítulo, apresenta a definição do problema que foi motivo de estudo e a motivação pela qual deu origem ao desenvolvimento do trabalho. Além do mais, apresenta os objetivos gerais e específicos definidos para que o projeto em questão seja realizado. Por fim, discorre sobre estado da arte do presente tema.

No segundo capítulo são apresentados os fundamentos, que consistem na revisão de literatura e fundamentação teórica, esse tem por objetivo histórica e teórica do trabalho. 

O terceiro capítulo consiste em apresentar o sistema mecatrônico, descrevendo a parte estrutural do protótipo. É neste capítulo também que é realizado as justificativas dos materiais escolhidos, suas características. Ao final, é apresentado a tabela de custo total.

O capítulo seguinte, o quarto deste trabalho, é encontrado o modelo não linear para o pêndulo sobre duas rodas. Em seguida, é feito a linearização e discretização do modelo para que o projeto do controlador seja possível. Em seguida, em ambiente de simulação, implementa o controlador ao sistema não linear e apresenta os resultados obtidos.

O quinto capítulo é dedicado basicamente a implementação do controlador projetado a planta real.

Por último, no capítulo de considerações finais, é descrito as conclusões deste trabalho. Detalha as dificuldades encontradas para realização do mesmo e propostas futuras.



