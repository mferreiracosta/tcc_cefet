%================================================================================================
%=========================================== CAPA 1 =============================================
%================================================================================================
\vspace*{1.0cm}

\begin{center}
{\sc  Centro Federal de Educação Tecnológica de Minas Gerais\\ {\em Campus} Divinópolis\\ Graduação em Engenharia Mecatrônica}
\end{center}

\vspace*{3.0cm}

\begin{center}
\large \autor % alterar também mais abaixo.
\end{center}


\vspace*{2.5cm}

\begin{center}
{\sc  \titulo} % alterar também mais abaixo.
\end{center}

\vspace*{4cm}

\epsfxsize=0.175\columnwidth
\centerline{\epsffile{logocefet}}

\null\vfill

\begin{center}
\cidade\\\ano % alterar também mais abaixo.
\end{center}


%Não numera a página:
\thispagestyle{empty}

%================================================================================================
%=========================================== CAPA 2 =============================================
%================================================================================================
\newpage
\vspace*{1.2cm}

\begin{center}
\large \autor % alterar também mais acima.
\end{center}


\vspace*{3.0cm}

\begin{center}
{\sc \titulo}  % alterar também mais acima.
\end{center}

\vspace*{2.5cm}

\begin{flushright}
\begin{minipage}{9.0cm}
Monografia de Trabalho de Conclusão de Curso apresentada ao
Colegiado de Graduação em Engenharia Mecatrônica
como  parte dos requisitos exigidos para a obtenção do título de
Engenheiro Mecatrônico.  \\
Áreas de integração: {Controle e Mecânica}.

\vspace*{1cm}

Orientador: \orientador \\
%Co-orientador: \coorientador
\end{minipage}
\end{flushright}

\vspace*{2.5cm}

%\epsfxsize=0.175\columnwidth
%\centerline{\epsffile{logocefet}} % Aqui poderia ser substituido por um logo do curso ou do campus

\null\vfill

\begin{center}
\cidade\\\ano  % alterar também mais acima.
\end{center}


%Não numera a página:
\thispagestyle{empty}

%================================================================================================
%=========================================== CAPA 3 =============================================
%================================================================================================
\newpage

\vspace*{1.2cm}
\begin{center}
\large \autor\\%[-1mm] % se for preciso ajustar a altura do texto
\end{center}


\vspace*{1.8cm}

\begin{center}
{\sc  \titulo}
\end{center}

\vspace*{2.25cm}

\begin{flushright}
\begin{minipage}{9.0cm}
Monografia de Trabalho de Conclusão de Curso apresentada ao
Colegiado de Graduação em Engenharia Mecatrônica
como  parte dos requisitos exigidos para a obtenção do título de
Engenheiro Mecatrônico.  \\
Áreas de integração: {Controle e Mecânica}.\\


\vspace*{0.5cm}

\end{minipage}
\end{flushright}

\vspace*{2cm}

\noindent Comissão Avaliadora:
\vspace*{0.5cm}

\noindent \begin{minipage}{0.5\linewidth}
Prof. Dr. Luís Filipe Pereira Silva \\[1mm]
CEFET/MG {\em Campus} Divinópolis
\end{minipage}
\begin{minipage}{0.5\linewidth}
Prof. Dr. Jean Carlos Pereira \\[1mm]
CEFET/MG {\em Campus} Divinópolis
\end{minipage}

\vspace*{0.4cm}

\noindent \begin{minipage}{0.5\linewidth}
\noindent  Prof. Dr. Lucas Silva de Oliveira \\[1mm]
\noindent  CEFET/MG {\em Campus} Divinópolis
\end{minipage}

\null \vfill

\begin{center}
\cidade\\\ano
\end{center}

%Não numera a página:
\thispagestyle{empty}
%======================================== Dedicatória =========================================
\newpage
\chapter*{}
\null
\vfill
\begin{flushright}
\begin{minipage}{6.5cm}
{\sc Ao meu pai.}
\end{minipage} \\[8mm]
\end{flushright}

%==============================================================================================

%======================================= Agradecimentos =======================================
\newpage
\chapter*{Agradecimentos}


\noindent Agradeço,\\[2mm]
\begin{itemize}
    \item aos meus avós João Custódio (vô Ló) e Maria das Dores (vó Lia), que sempre cuidaram de mim, da minha mãe e meus irmãos quando mais precisamos. Além de tudo, agradeço por me apoiarem e terem feito eu seguir o caminho dos estudos.
    \item à minha mãe Cintia Aparecida e meu padrasto Roni, por sempre me  apoiarem, ajudarem e serem os exemplos ao qual me espelho. Eu amo vocês incondicionalmente.
    \item à minha namorada Ruanna, pelo companheirismo inigualável e por sempre acreditar em mim e me fazer acreditar que sou capaz o tempo todo. Com você eu tenho certeza e clareza dos meus objetivos. Obrigado por aguentar todos esses anos de estudos ao meu lado e sempre me apoiar nesse meu sonho, amo você.
    \item ao Professor Dr. Luís Filipe Pereira Silva, pelo suporte, orientação e pelo compartilhamento de conhecimento para este trabalho e outros para a vida como a ``bastter.com".
    \item ao professor Dr. Jean Carlos Pereira, que me identifiquei bastante e pude trabalhar em projetos de pesquisa ao longo do curso, inclusive este que antes de se tornar um trabalho final de curso foi tema de pesquisa.
    \item aos meus irmãos Ana Paula e Yuri, que são tudo para mim. Eu, por ser o mais velho, tenho o sentimento de sempre cuidar de vocês, então é isso que sempre tento fazer. E por mais que temos nossas brigas de irmãos, sou muito sortudo por ter vocês em minha vida.
    \item à turma 8, por todos os ``burgões", pelos momentos de descontração e pelas amizades que levarei para sempre no coração. Vocês tornaram a caminhada mais leve e divertida.
    \item ao meu companheiro Delgado, principalmente, durante este trabalho. Foram algumas reuniões bem longas após o fim de trabalho e você sempre se prontificou a me ajudar.
    \item ao meu amigo de longa data Mateus Pereira, em especial, por ter emprestado sua fresa e seu galpão mecânico para construção da estrutura física da planta.
    \item aos meus colegas de turma, Duda, Maria Vitória e Kesley, que hoje está se tornando uma grande amizade. Obrigado por confiarem que eu seria capaz, me apoiarem e até por emprestarem uma ponte H aos 45 do segundo tempo.
    \item ao corpo docente e funcionário do CEFET-MG \textit{Campus} Divinópolis, pela colaboração e atenção.

\end{itemize}

\vfill
%==============================================================================================
%======================================== Frase Miss Brasil ===================================
\newpage
\chapter*{}
\null\vfill
\begin{flushright}
\begin{minipage}{9.0cm}
Que todos os nossos esforços estejam sempre focados no desafio à impossibilidade. Todas as grandes conquistas humanas vieram daquilo que parecia impossível.
\end{minipage}
\end{flushright}


\begin{flushright}
 \href{https://viacarreira.com/epigrafe-para-tcc/}{Charles Chaplin}
\end{flushright}

%==============================================================================================

