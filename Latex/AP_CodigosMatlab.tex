\chapter{Códigos}\label{Apendice:CodigoMatlab}
%\pagenumbering{arabic}

\begin{minted}[breaklines=true]{C}

Link para códigos e modelagem completa do trabalho: https://github.com/mferreiracosta/tcc_cefet

%% PROGRAMA PÊNDULO INVERTIDO - TCC 2 - 2022.1
% Autor: Matheus Ferreira Costa
% Orientador: Luís Filipe 

clc; clear all; close all

%% Definição dos Parâmetros
syms g M m L r R kt

g = 9.80665;
M = 0.066;    
m = 0.788;
L = 0.18;    r = 0.033;
R = 0.92;    kt = 0.54;

den = 2*M*r^2 + m*(r^2 + L^2 + 2*L*r);

%% Calculando as matrizes do sistema A, B, C e D

A = [      0      1;
     (m*g*L)/den  0];
B = [0; kt/(R*den)];
C = [1 0];
D = [0];

sys = ss(A,B,C,D);

%% Calculando o rank de controlabilidade e observabilidade do sistema
Sc = ctrb(sys);
So = obsv(sys);

rank_ctrb = rank(Sc);
rank_obsv = rank(So);

%% Escolhendo as matrizes Q e R para calcular o controlador LQR
Q = [1   0;            
     0   1];
R = 1;

%% Construção do Sistema em Espaço de Estados no Tempo Discreto
Ts = 0.05;
sysd = c2d(sys,Ts);

% Matrizes discretas: Ad e Bd, Cd = C e Dd = D
Ad = sysd.a;
Bd = sysd.b;

%% Construção do controlador LQR no Tempo Discreto
[Krd,S,e] = dlqr(Ad,Bd,Q,R);

%% Construção do LQE - Filtro de Kalman no Tempo Discreto
Qo = 1;                  % matriz de covariância do distúrbio           
Ro = 1;                  % (escalar) covariância do ruído de medição  

% Construção do LQE - Filtro de Kalman
[kalmf,Lkalm,P] = kalman(sysd,Qo,Ro);

Ld = dlqr(Ad',C',Qo,Ro);
Ld = Ld';

sysKFd = ss(Ad-Ld*C,[Bd Ld],eye(2),zeros(2,2));

\end{minted}