%================================================================================================
%=========================================== Folha 1 =============================================
%================================================================================================
\vspace*{1.0cm}

\begin{center}
{\sc  Centro Federal de Educação Tecnológica de Minas Gerais\\
{\em Campus} Divinópolis\\
Graduação em Engenharia Mecatrônica}
\end{center}

\vspace*{3.0cm}

\begin{center}
\large \autor % alterar também mais abaixo.
\end{center}


\vspace*{2.5cm}

\begin{center}
{\sc  \titulo} % alterar também mais abaixo.
\end{center}

\vspace*{4cm}

%\epsfxsize=0.175\columnwidth
%\centerline{\epsffile{logocefet}}

\null\vfill

\begin{center}
Divinópolis\\\ano % alterar também mais abaixo.
\end{center}

%Não numera a página:
\thispagestyle{empty}

\newpage
%================================================================================================
%=========================================== Folha 2 =============================================
%================================================================================================

\vspace*{1.2cm}

\begin{center}
\large \autor % alterar também mais acima.
\end{center}


\vspace*{3.0cm}

\begin{center}
{\sc  \titulo}  % alterar também mais acima.
\end{center}

\vspace*{2.5cm}

\begin{flushright}
\begin{minipage}{9.0cm}
Relatório parcial de Trabalho de Conclusão de Curso apresentado à
comissão avaliadora do curso de Graduação
em Engenharia Mecatrônica
como  parte dos requisitos exigidos para a obtenção da aprovação
na disciplina de TCC I.  \\
Áreas de integração: {coloque aqui os nomes das áreas envolvidas}.

\vspace*{1cm}

Orientador: \orientador \\
Co-orientador: Nome do co-orientador
\end{minipage}
\end{flushright}

\vspace*{2.5cm}

%\epsfxsize=0.175\columnwidth
%\centerline{\epsffile{logocefet}} % Aqui poderia ser substituido por um logo do curso ou do campus

\null\vfill

\begin{center}
Divinópolis\\\ano  % alterar também mais acima.
\end{center}

%Não numera a página:
\thispagestyle{empty}

%================================================================================================
%=========================================== Folha 3 =============================================
%================================================================================================
\newpage

\vspace*{1.2cm}
\begin{center}
\large \autor\\%[-1mm] % se for preciso ajustar a altura do texto
\end{center}


\vspace*{1.8cm}

\begin{center}
{\sc  \titulo}
\end{center}

\vspace*{2.25cm}

\begin{flushright}
\begin{minipage}{9.0cm}
Relatório parcial de Trabalho de Conclusão de Curso apresentado à
comissão avaliadora do curso de Graduação em Engenharia Mecatrônica
como  parte dos requisitos exigidos para a obtenção da aprovação
na disciplina de TCC I.

Eixos de formação: {coloque aqui os nomes dos eixos envolvidos}.
% Acrescentar na versão final de TCC II: Aprovada pela banca examinadora
% em defesa pública realizada no dia xx de mês de ano.

\vspace*{0.5cm}

\end{minipage}
\end{flushright}

\vspace*{2cm}

\noindent Comissão Avaliadora:
\vspace*{0.5cm}

\noindent \begin{minipage}{0.5\linewidth}
Prof. Dr. Fulano de tal \\[1mm]
CEFET/MG {\em Campus} II
\end{minipage}
\begin{minipage}{0.5\linewidth}
Prof. Dr.Beltrano de XYZ \\[1mm]
Departamento/Instituição
\end{minipage}

\vspace*{0.4cm}

\noindent \begin{minipage}{0.5\linewidth}
\noindent  Prof. M. Sc. Fulano2 \\[1mm]
\noindent  Departamento/Instituição
\end{minipage}
\begin{minipage}{0.5\linewidth}
Prof. Esp. Fulano 3\\[1mm]
Departamento/Instituição
\end{minipage}


\null \vfill

\begin{center}
Divinópolis \\\ano
\end{center}

%Não numera a página:
\thispagestyle{empty}



% Pendências: 1) numeração em romanos começa a aparecer a partir da página
%da dedicatória (pag. 5), inclusive; 2) fazer a versão TCC II